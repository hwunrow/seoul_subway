%%%%%%%%%%%%%%%%%%%%%%%%%%%%%%%%%%%%%%%%%
% Programming/Coding Assignment
% LaTeX Template
%
% This template has been downloaded from:
% http://www.latextemplates.com
%
% Original author:
% Ted Pavlic (http://www.tedpavlic.com)
%
% Note:
% The \lipsum[#] commands throughout this template generate dummy text
% to fill the template out. These commands should all be removed when 
% writing assignment content.
%
% This template uses a Perl script as an example snippet of code, most other
% languages are also usable. Configure them in the "CODE INCLUSION 
% CONFIGURATION" section.
%
%%%%%%%%%%%%%%%%%%%%%%%%%%%%%%%%%%%%%%%%%

%----------------------------------------------------------------------------------------
%	PACKAGES AND OTHER DOCUMENT CONFIGURATIONS
%----------------------------------------------------------------------------------------

\documentclass{article}

\usepackage{fancyhdr} % Required for custom headers
\usepackage{lastpage} % Required to determine the last page for the footer
\usepackage{extramarks} % Required for headers and footers
\usepackage[usenames,dvipsnames]{color} % Required for custom colors
\usepackage{graphicx} % Required to insert images
\usepackage{listings} % Required for insertion of code
\usepackage{courier} % Required for the courier font
\usepackage{enumerate} %Required to create lists
\usepackage{amsmath}
\usepackage{amssymb} %Required for math symbols

% Margins
\topmargin=-0.45in
\evensidemargin=0in
\oddsidemargin=0in
\textwidth=6.5in
\textheight=9.0in
\headsep=0.25in

\linespread{1.1} % Line spacing

% Set up the header and footer
\pagestyle{fancy}
\lhead{\hmwkAuthorName} % Top left header
\chead{\hmwkClass\ (\hmwkClassInstructor\ \hmwkClassTime): \hmwkTitle} % Top center head
\rhead{\firstxmark} % Top right header
\lfoot{\lastxmark} % Bottom left footer
\cfoot{} % Bottom center footer
\rfoot{Page\ \thepage\ of\ \protect\pageref{LastPage}} % Bottom right footer
\renewcommand\headrulewidth{0.4pt} % Size of the header rule
\renewcommand\footrulewidth{0.4pt} % Size of the footer rule

\setlength\parindent{0pt} % Removes all indentation from paragraphs

%----------------------------------------------------------------------------------------
%	CODE INCLUSION CONFIGURATION
%----------------------------------------------------------------------------------------

\definecolor{MyDarkGreen}{rgb}{0.0,0.4,0.0} % This is the color used for comments
\lstloadlanguages{Perl} % Load Perl syntax for listings, for a list of other languages supported see: ftp://ftp.tex.ac.uk/tex-archive/macros/latex/contrib/listings/listings.pdf
\lstset{language=Java, 
        frame=single, % Single frame around code
        basicstyle=\small\ttfamily, % Use small true type font
        keywordstyle=[1]\color{Blue}\bf, % Perl functions bold and blue
        keywordstyle=[2]\color{Purple}, % Perl function arguments purple
        keywordstyle=[3]\color{Blue}\underbar, % Custom functions underlined and blue
        identifierstyle=, % Nothing special about identifiers                                         
        commentstyle=\usefont{T1}{pcr}{m}{sl}\color{MyDarkGreen}\small, % Comments small dark green courier font
        stringstyle=\color{Purple}, % Strings are purple
        showstringspaces=false, % Don't put marks in string spaces
        tabsize=5, % 5 spaces per tab
        morecomment=[l][\color{Blue}]{...}, % Line continuation (...) like blue comment
        numbers=left, % Line numbers on left
        firstnumber=1, % Line numbers start with line 1
        numberstyle=\tiny\color{Blue}, % Line numbers are blue and small
        stepnumber=5 % Line numbers go in steps of 5
}

% Creates a new command to include a perl script, the first parameter is the filename of the script (without .pl), the second parameter is the caption
\newcommand{\javascript}[2]{
\begin{itemize}
\item[]\lstinputlisting[caption=#2,label=#1]{#1.pl}
\end{itemize}
}

%----------------------------------------------------------------------------------------
%	DOCUMENT STRUCTURE COMMANDS
%	Skip this unless you know what you're doing
%----------------------------------------------------------------------------------------

% Header and footer for when a page split occurs within a problem environment
\newcommand{\enterProblemHeader}[1]{
\nobreak\extramarks{#1}{#1 continued on next page\ldots}\nobreak
\nobreak\extramarks{#1 (continued)}{#1 continued on next page\ldots}\nobreak
}

% Header and footer for when a page split occurs between problem environments
\newcommand{\exitProblemHeader}[1]{
\nobreak\extramarks{#1 (continued)}{#1 continued on next page\ldots}\nobreak
\nobreak\extramarks{#1}{}\nobreak
}

\setcounter{secnumdepth}{0} % Removes default section numbers
\newcounter{homeworkProblemCounter} % Creates a counter to keep track of the number of problems

\newcommand{\homeworkProblemName}{}
\newenvironment{homeworkProblem}[1][Problem \arabic{homeworkProblemCounter}]{ % Makes a new environment called homeworkProblem which takes 1 argument (custom name) but the default is "Problem #"
\stepcounter{homeworkProblemCounter} % Increase counter for number of problems
\renewcommand{\homeworkProblemName}{#1} % Assign \homeworkProblemName the name of the problem
\section{\homeworkProblemName} % Make a section in the document with the custom problem count
\enterProblemHeader{\homeworkProblemName} % Header and footer within the environment
}{
\exitProblemHeader{\homeworkProblemName} % Header and footer after the environment
}

\newcommand{\problemAnswer}[1]{ % Defines the problem answer command with the content as the only argument
\noindent\framebox[\columnwidth][c]{\begin{minipage}{0.98\columnwidth}#1\end{minipage}} % Makes the box around the problem answer and puts the content inside
}

\newcommand{\homeworkSectionName}{}
\newenvironment{homeworkSection}[1]{ % New environment for sections within homework problems, takes 1 argument - the name of the section
\renewcommand{\homeworkSectionName}{#1} % Assign \homeworkSectionName to the name of the section from the environment argument
\subsection{\homeworkSectionName} % Make a subsection with the custom name of the subsection
\enterProblemHeader{\homeworkProblemName\ [\homeworkSectionName]} % Header and footer within the environment
}{
\enterProblemHeader{\homeworkProblemName} % Header and footer after the environment
}

%----------------------------------------------------------------------------------------
%	NAME AND CLASS SECTION
%----------------------------------------------------------------------------------------

\newcommand{\hmwkTitle}{Assignment\ \#5} % Assignment title
\newcommand{\hmwkDueDate}{Friday,\ December\ 12,\ 2014} % Due date
\newcommand{\hmwkClass}{CSED\ 233} % Course/class
\newcommand{\hmwkClassTime}{12:30pm} % Class/lecture time
\newcommand{\hmwkClassInstructor}{BoHyung Han} % Teacher/lecturer
\newcommand{\hmwkAuthorName}{Newell Wunrow} % Your name

%----------------------------------------------------------------------------------------
%	TITLE PAGE
%----------------------------------------------------------------------------------------

\title{
\vspace{2in}
\textmd{\textbf{\hmwkClass:\ \hmwkTitle}}\\
\normalsize\vspace{0.1in}\small{Due\ on\ \hmwkDueDate}\\
\vspace{0.1in}\large{\textit{\hmwkClassInstructor\ \hmwkClassTime}}
\vspace{3in}
}

\author{\textbf{\hmwkAuthorName}}
\date{} % Insert date here if you want it to appear below your name

%----------------------------------------------------------------------------------------

\begin{document}

\maketitle
\clearpage
%----------------------------------------------------------------------------------------
%	TABLE OF CONTENTS
%----------------------------------------------------------------------------------------

%\setcounter{tocdepth}{1} % Uncomment this line if you don't want subsections listed in the ToC

%\newpage
%\tableofcontents
%\newpage
% Uncomment if you want to have a TOC
%----------------------------------------------------------------------------------------
%	Explanation of Problem and Solution
%----------------------------------------------------------------------------------------

% To have just one problem per page, simply put a \clearpage after each problem
\title{
\textmd{\textbf{Problem and Solution}}
}
\newline 
I will give a brief explanation of the problem and my solution.
\newline
Firstly, the assignment was to find the shortest path between two subway staitons in the Seoul subway system, and then to output the minimum distance,the transfer stations and the distances between them. A transfer stations is defined to be a station where you switch from one line to the other. 
\newline
In short, I solved this problem by first organizing the data given to create a graph where the vertices represented subway stations and the edges represented the railways between each station. One key data manipulation I did was that I added an edge with weight 0 at each transfer station. For example, if you wanted to transfer from line 3 to line 2 at 교대 you would travel along the edge from station 0330 to station 0224. I then used Dijkstra's Algorithm to find the minimum distance path between the given source and destination stations.

\clearpage
%----------------------------------------------------------------------------------------
%	Explanation of Implementaiton
%----------------------------------------------------------------------------------------
\title{
\textmd{\textbf{Explanation of Implementaiton}}
}
\newline

Here is a brief explanation of my implementations.

\begin{center} SeoulSubway Class \end{center}
This is the main class which reads the arguments given in the command prompt. First, we parse the data given the text files by utilizing the method already provided in the ReadSubwayData class. Next for each row in the text files we create an object of type edge and vertice. As stated on the previous page, I then inserted edges at transfer stations with length 0. I then created a graph that represents the Seoul subway station system using these two lists of edges and vertices. Next for each edge in the graph, we assign a start and finish vertex in the list vertices. We then call the dijkstra function to find the minimum path between our source and destination stations. Finally, we output our results in the format as stated in the assignment pdf.

\begin{center} Edge Class \end{center}
The edge class I created has 8 variables.
 
\begin{description}
	\item[line]: represents the line the railway is on
	\item[station name1]:  the name of the start station
	\item[station code1]: the code of the start station
	\item[station name2]: the name of the finish station
	\item[station code2]: the code of the finish station
	\item[weight]: the distance between the two stations
	\item[start]: the vertex associated with the start station
	\item[finish]: the vertex associated with the finish station
\end{description}
Within the class there are basic constructor, mutator, and accessor methods.

\begin{center} Vertice Class \end{center}
The vertice class has 7 variables.
\begin{description}
	\item[line]: the line the station is on
	\item[name]: the name of the station
	\item[code]: the code of the station
	\item[lat]: the longitude of the station, was not used in the program at all
	\item[lon]: the latitude of the station, was not used in the program at all
	\item[distance]: the distance from the source station, used in Dijkstra's Algorithm
	\item[previous]: the station visited previous in a path, used in Dijkstra's Algorithm
\end{description}
Within the class there are constructor, mutator, and accessor methods.
\clearpage

\begin{center} Graph Class \end{center}
The Graph class has two data fields.
\begin{description}
	\item[Vertices]: list of stations in the subway graph
	\item[Edges]: holds the railways between stations in the subway graph	
\end{description}

The functions in this class include contrsuctor methods and the following:
\begin{description}
	\item[findVertices(String c)]: \\ \hfill this searches the list of vertices in the graph for a vertec with a specific code c
	\item[findVertices(String c, List V)] :\\ \hfill  this searches a specific list V for a vertex with a specific code c
	\item[isEmpty( )]: \\ \hfill returns true if there is at least one vertex in the graph, otherwise fale
	\item[getEdgeDistance(Vertice a, Vertice b)]: \\ \hfill returns the distance between two adjacent vertices
	\item[incidentVertice(Vertice v)] : \\ \hfill returns a list of all the incident stations of a station, if it is a transfer station this includes all the incident stations on other subway lines, it was used in Dijkstra's Algorithm
	\item[removeMin(List V)] : \\ \hfill removes the minimum distance vertex from the graph and returns it, it was used in Dijkstra's Algorithm
	\item[replaceKey(Vertice v, double d, List  V)]: \\ \hfill sets the distance for a specific node in a list of nodes,, it was used in Dijkstra's Algorithm
	\item[transferfy(List V)] : \\ \hfill returns the list of transfer stations along a path, it also calculates the distance between each transfer station and assigns it to the distance variable of the starting station, it was used in Dijkstra's Algorithm
	\item[displayVertice]: \\ \hfill displays the line, code and distance information for each vertex in a list, this was used A LOT for debugging purposes
\end{description}
\clearpage

\begin{center} Dijkstra's Algorithm  \end{center}
Here is a brief description of Dijkstra's algorithm.
\begin{enumerate}
	\item Set the distance at the starting vertex to 0 and each vertex to inifinity.
	\item Remove the vertex with the smallest distance.
	\item For all the incident vertices  v to the removed node, if $distance_{edge} + distance_{removed} < distance_v$, 				then update the distance.
	\item In order to keep track of the path, if the distance of a vertex is updated set the previous vertex to the removed minimum 
		distance vertex.
\end{enumerate}

After completing this, you have a minimum spanning tree of the graph. (aka the shortest path from the starting vertex to all other vertices denoted by $path$ ). Next I defined a new path $finishPath$ with minimum distance from the starting to ending vertices by backtracking along $path$.

Finally, in compliance with the assignment. I created a list of the transfer vertices visited along $finishPath$. To do this I simply backtracked along $finishPath$ until I found a vertex where its previous vertex had a different line. To get the distance between the transfer vertices I simply use the Dijkstra distance (denoted by $distance_{i}$ at each vertice with the following formula:
$$ transferDistance_{a,b} = distance_{b} - distance_{a}$$

We now have our desired list of transfer stations and distances between them for the output.

A copy of my code is in the appendix for reference.

\clearpage
%----------------------------------------------------------------------------------------
%	Appendix
%----------------------------------------------------------------------------------------
\title{
\textmd{\textbf{Appendix}}
}
\newline


Listing \ref{SeoulSubway}
\javascript{SeoulSubway}{SeoulSubway}

Listing \ref{Graph}
\javascript{Graph}{Graph}


%----------------------------------------------------------------------------------------

\end{document}